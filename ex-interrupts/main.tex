%----------------------------------------------------------------------------------------
%   PACKAGES AND OTHER DOCUMENT CONFIGURATIONS
%----------------------------------------------------------------------------------------

\documentclass[18pt]{extarticle}

%%%%%%%%%%%%%%%%%%%%%%%%%%%%%%%%%%%%%%%%%
% Lachaise Assignment
% Structure Specification File
% Version 1.0 (26/6/2018)
%
% This template originates from:
% http://www.LaTeXTemplates.com
%
% Authors:
% Marion Lachaise & François Févotte
% Vel (vel@LaTeXTemplates.com)
%
% License:
% CC BY-NC-SA 3.0 (http://creativecommons.org/licenses/by-nc-sa/3.0/)
% 
%%%%%%%%%%%%%%%%%%%%%%%%%%%%%%%%%%%%%%%%%

%----------------------------------------------------------------------------------------
%	PACKAGES AND OTHER DOCUMENT CONFIGURATIONS
%----------------------------------------------------------------------------------------

\setlength{\parindent}{0em}
\setlength{\parskip}{.5em}

\usepackage[main=greek,english]{babel}

\usepackage{amsmath,amsfonts,stmaryrd,amssymb} % Math packages

\usepackage{enumerate} % Custom item numbers for enumerations

\usepackage[ruled]{algorithm2e} % Algorithms

\usepackage[framemethod=tikz]{mdframed} % Allows defining custom boxed/framed environments

\usepackage{enumitem}

\usepackage{listings} % File listings, with syntax highlighting

\lstset{
	basicstyle=\ttfamily, % Typeset listings in monospace font
}

\usepackage{hyperref}
\hypersetup{
    colorlinks=true,
    linkcolor=blue,
    filecolor=magenta,      
    urlcolor=cyan,
}

\usepackage{fancyhdr}

\usepackage{float}

\usepackage{adjustbox}

\usepackage{tikz}


%----------------------------------------------------------------------------------------
%	DOCUMENT MARGINS
%----------------------------------------------------------------------------------------

\usepackage{geometry} % Required for adjusting page dimensions and margins

\geometry{
	paper=a4paper, % Paper size, change to letterpaper for US letter size
	top=2.5cm, % Top margin
	bottom=3cm, % Bottom margin
	left=2.5cm, % Left margin
	right=2.5cm, % Right margin
	headheight=14pt, % Header height
	footskip=1.5cm, % Space from the bottom margin to the baseline of the footer
	headsep=1.2cm, % Space from the top margin to the baseline of the header
	%showframe, % Uncomment to show how the type block is set on the page
}

%----------------------------------------------------------------------------------------
%	FONTS
%----------------------------------------------------------------------------------------

\usepackage[utf8]{inputenc} % Required for inputting international characters
\usepackage[T1]{fontenc} % Output font encoding for international characters

\usepackage{XCharter} % Use the XCharter fonts

\usepackage{csquotes}
\usepackage{dirtytalk}

%----------------------------------------------------------------------------------------
%	BIBLIOGRAPHY
%----------------------------------------------------------------------------------------

\usepackage[backend=bibtex]{biblatex}


%----------------------------------------------------------------------------------------
%	COMMAND LINE ENVIRONMENT
%----------------------------------------------------------------------------------------

% Usage:
% \begin{commandline}
%	\begin{verbatim}
%		$ ls
%		
%		Applications	Desktop	...
%	\end{verbatim}
% \end{commandline}

\mdfdefinestyle{commandline}{
	leftmargin=10pt,
	rightmargin=10pt,
	innerleftmargin=15pt,
	middlelinecolor=black!50!white,
	middlelinewidth=2pt,
	frametitlerule=false,
	backgroundcolor=black!5!white,
	frametitle={Command Line},
	frametitlefont={\normalfont\sffamily\color{white}\hspace{-1em}},
	frametitlebackgroundcolor=black!50!white,
	nobreak,
	singleextra={%
    }
}

% Define a custom environment for command-line snapshots
\newenvironment{commandline}{
	\medskip
	\begin{mdframed}[style=commandline]
}{
	\end{mdframed}
	\medskip
}

%----------------------------------------------------------------------------------------
%	FILE CONTENTS ENVIRONMENT
%----------------------------------------------------------------------------------------

% Usage:
% \begin{file}[optional filename, defaults to "File"]
%	File contents, for example, with a listings environment
% \end{file}

\mdfdefinestyle{file}{
	innertopmargin=1.6\baselineskip,
	innerbottommargin=0.8\baselineskip,
	topline=false, bottomline=false,
	leftline=false, rightline=false,
    leftmargin=0cm,rightmargin=0cm,
	singleextra={%
		\draw[fill=black!10!white](P)++(0,-1.2em)rectangle(P-|O);
		\node[anchor=north west]
		at(P-|O){\ttfamily\mdfilename};
		%
		\def\l{3em}
		\draw(O-|P)++(-\l,0)--++(\l,\l)--(P)--(P-|O)--(O)--cycle;
		\draw(O-|P)++(-\l,0)--++(0,\l)--++(\l,0);
	},
	firstextra={%
		\draw[fill=black!10!white](P)++(0,-1.2em)rectangle(P-|O);
		\node[anchor=north west]
		at(P-|O){\ttfamily\mdfilename};
		%
	},
    middleextra={%
    },
    secondextra={%
		\def\l{3em}
		\draw(O-|P)++(-\l,0)--++(\l,\l)--(P)--(P-|O)--(O)--cycle;
		\draw(O-|P)++(-\l,0)--++(0,\l)--++(\l,0);
    },
    nobreak,
}

% Define a custom environment for file contents
\newenvironment{file}[1][File]{ % Set the default filename to "File"
	\medskip
	\newcommand{\mdfilename}{#1}
	\begin{mdframed}[style=file]
}{
	\end{mdframed}
	\medskip
}

%----------------------------------------------------------------------------------------
%	NUMBERED QUESTIONS ENVIRONMENT
%----------------------------------------------------------------------------------------

% Usage:
% \begin{question}[optional title]
%	Question contents
% \end{question}

\mdfdefinestyle{question}{
	innertopmargin=1.2\baselineskip,
	innerbottommargin=0.8\baselineskip,
	roundcorner=5pt,
	singleextra={%
		\draw(P-|O)node[xshift=1em,anchor=west,fill=white,draw,rounded corners=5pt]{%
		Άσκηση \theQuestion\questionTitle};
	},
	firstextra={%
		\draw(P-|O)node[xshift=1em,anchor=west,fill=white,draw,rounded corners=5pt]{%
		Άσκηση \theQuestion\questionTitle};
	},
}

\newcounter{Question} % Stores the current question number that gets iterated with each new question

% Define a custom environment for numbered questions
\newenvironment{question}[1][\unskip]{
	\bigskip
	\stepcounter{Question}
	\newcommand{\questionTitle}{~#1}
	\begin{mdframed}[style=question]
}{
	\end{mdframed}
	\medskip
}

%----------------------------------------------------------------------------------------
%	WARNING TEXT ENVIRONMENT
%----------------------------------------------------------------------------------------

% Usage:
% \begin{warn}[optional title, defaults to "Warning:"]
%	Contents
% \end{warn}

\mdfdefinestyle{warning}{
	topline=false, bottomline=false,
	leftline=false, rightline=false,
	nobreak,
	singleextra={%
		\draw(P-|O)++(-0.5em,0)node(tmp1){};
		\draw(P-|O)++(0.5em,0)node(tmp2){};
		\fill[black,rotate around={45:(P-|O)}](tmp1)rectangle(tmp2);
		\node at(P-|O){\color{white}\scriptsize\bf !};
		\draw[very thick](P-|O)++(0,-1em)--(O);%--(O-|P);
	}
}

% Define a custom environment for warning text
\newenvironment{warn}[1][Warning:]{ % Set the default warning to "Warning:"
	\medskip
	\begin{mdframed}[style=warning]
		\noindent{\textbf{#1}}
}{
	\end{mdframed}
}

%----------------------------------------------------------------------------------------
%	INFORMATION ENVIRONMENT
%----------------------------------------------------------------------------------------

% Usage:
% \begin{info}[optional title, defaults to "Info:"]
% 	contents
% 	\end{info}

\mdfdefinestyle{info}{%
	topline=false, bottomline=false,
	leftline=false, rightline=false,
	nobreak,
	singleextra={%
		\fill[black](P-|O)circle[radius=0.4em];
		\node at(P-|O){\color{white}\scriptsize\bf i};
		\draw[very thick](P-|O)++(0,-0.8em)--(O);%--(O-|P);
	}
}

% Define a custom environment for information
\newenvironment{info}[1][Info:]{ % Set the default title to "Info:"
	\medskip
	\begin{mdframed}[style=info]
		\noindent{\textbf{#1}}
}{
	\end{mdframed}
}

\newcommand{\src}[1]{{\texttt{#1}}}


%----------------------------------------------------------------------------------------
%	EXTRA STYLES/SETTINGS
%----------------------------------------------------------------------------------------

\usetikzlibrary{trees}
\usetikzlibrary{fit}
\usetikzlibrary{shapes}
\usetikzlibrary{arrows.meta, positioning, shadows}
\usetikzlibrary{chains,shapes.multipart}
\usetikzlibrary{external}
\usetikzlibrary{matrix}
\usetikzlibrary{shapes.callouts}

\tikzstyle{rect} = [rectangle,fill=white, text centered, draw]
\tikzstyle{database} = [draw, cylinder, shape border rotate = 90, aspect = 0.2]
\tikzstyle{thick-arrow} = [->, thick]
\tikzstyle{arrow} = [thick,->,>=stealth]
\tikzstyle{arrow-small} = [->,>=stealth]
\tikzstyle{simple-rect} = [rectangle, text centered, draw = black, inner sep=4mm]
\tikzstyle{startstop} = [rectangle, rounded corners, minimum width=3cm, minimum height=1cm,text centered, draw=black, fill=red!30]
\tikzstyle{io} = [trapezium, trapezium left angle=70, trapezium right angle=110, minimum width=3cm, minimum height=1cm, text centered, draw=black, fill=blue!30]
\tikzstyle{process} = [rectangle, minimum width=3cm, minimum height=1cm, text centered, draw=black, fill=orange!30]
\tikzstyle{decision} = [diamond, minimum width=3cm, minimum height=1cm, text centered, draw=black, fill=green!30]
\tikzstyle{arrow} = [thick,->,>=stealth]
\tikzset{
    dcs/.style = {double copy shadow},
}

% Tikz shape for document-like icon
\makeatletter
\pgfdeclareshape{document}{
\inheritsavedanchors[from=rectangle] % this is nearly a rectangle
\inheritanchorborder[from=rectangle]
\inheritanchor[from=rectangle]{center}
\inheritanchor[from=rectangle]{north}
\inheritanchor[from=rectangle]{south}
\inheritanchor[from=rectangle]{west}
\inheritanchor[from=rectangle]{east}
% ... and possibly more
\backgroundpath{% this is new
% store lower right in xa/ya and upper right in xb/yb
\southwest \pgf@xa=\pgf@x \pgf@ya=\pgf@y
\northeast \pgf@xb=\pgf@x \pgf@yb=\pgf@y
% compute corner of ‘‘flipped page’’
\pgf@xc=\pgf@xb \advance\pgf@xc by-10pt % this should be a parameter
\pgf@yc=\pgf@yb \advance\pgf@yc by-10pt
% construct main path
\pgfpathmoveto{\pgfpoint{\pgf@xa}{\pgf@ya}}
\pgfpathlineto{\pgfpoint{\pgf@xa}{\pgf@yb}}
\pgfpathlineto{\pgfpoint{\pgf@xc}{\pgf@yb}}
\pgfpathlineto{\pgfpoint{\pgf@xb}{\pgf@yc}}
\pgfpathlineto{\pgfpoint{\pgf@xb}{\pgf@ya}}
\pgfpathclose
% add little corner
\pgfpathmoveto{\pgfpoint{\pgf@xc}{\pgf@yb}}
\pgfpathlineto{\pgfpoint{\pgf@xc}{\pgf@yc}}
\pgfpathlineto{\pgfpoint{\pgf@xb}{\pgf@yc}}
\pgfpathlineto{\pgfpoint{\pgf@xc}{\pgf@yc}}
}
}
\makeatother

\tikzstyle{doc}=[%
draw,
thick,
align=center,
color=black,
shape=document,
minimum width=20mm,
minimum height=28.2mm,
shape=document,
inner sep=2ex,
]

 % Include the file specifying the document structure and custom commands

%----------------------------------------------------------------------------------------
%   ASSIGNMENT INFORMATION
%----------------------------------------------------------------------------------------

\title{Υλοποίηση Διακοπών Λειτουργικού συστήματος (Interrupts) \\Λειτουργικά Συστήματα (ECE ΓΚ702)} % Title of the assignment

\author{
    \footnotesize Χρήστος Φείδας\\ 
    \footnotesize \src{fidas@upatras.gr} \and 
    \footnotesize Ευάγγελος Λάμπρου\\ 
    \footnotesize \src{e.lamprou@upnet.gr} \and
    \footnotesize Ζεϊμπέκης Βασίλειος\\
    \footnotesize \src{bzeimp@gmail.com}
} % Author name and email address

\date{University of Patras, Department of Electrical and Computer Engineering} % University, school and/or department name(s) and a date

\usepackage{fancyhdr}

%----------------------------------------------------------------------------------------

\begin{document}

\pagestyle{fancy}
%... then configure it.
\fancyhf{} % sets both header and footer to nothing
\renewcommand{\headrulewidth}{0pt}
\fancyhead{} % clear all header fields
\fancyfoot{} % clear all footer fields
\fancyfoot[L]{Χρήστος Φείδας, Ευάγγελος Λάμπρου, Ζεϊμπέκης Βασίλειος}
\fancyfoot[R]{\thepage}

\maketitle

%----------------------------------------------------------------------------------------
%   INTRODUCTION
%----------------------------------------------------------------------------------------

\section{Εισαγωγή}

Η εργασία αυτή κάνει μία εισαγωγή στην υλοποίηση των Interrupts και του Interrupt Descriptor Table
για το open source εκπαιδευτικό λειτουργικό σύστημα Skyl-OS.

Μπορείτε να βρείτε το συνοδευτικό βιβλίο του Skyl-OS \href{https://github.com/Billyzeim/Skyl-OS/blob/main/docs/book/main.pdf}{εδώ}.

\subsection{Ο πηγαίος κώδικας του Skyl-OS kernel.}

Ο πηγαίος κώδικας του SKyl-OS είναι διαθέσιμος ελεύθερα. 
Κάποιος μπορεί να κατεβάσει τον κώδικα, να τον επεξεργαστεί και να δημιουργήσει δικές του εκδόσεις του SKyl-OS.
Στα πλαίσια των εργασιών θα ζητηθεί να τροποποιήσετε κομμάτια του πυρήνα του Skyl-OS.

Για να εξερευνήσετε τον πηγαίο κώδικα του Skyl-OS μπορείτε να κάνετε λήψη του κώδικα 
από το \href{https://github.com/Billyzeim/Skyl-OS/blob/main/}{GitHub repo} και χρήση του αγαπημένου σας editor 
(Visual Studio Code, Vim, Emacs ...) για αναζήτηση σε αυτόν.

\subsection{Ο kernel του Skyl-OS}

Ο πυρήνας (kernel) του Skyl-OS, όπως των Linux, είναι μονολιθικός, πράγμα που σημαίνει πως αν κάποιος θέλει να 
προσθέσει νέα λειτουργικότητα σε αυτόν πρέπει να επεξεργαστεί τον πηγαίο κώδικα, 
να τον επαναμεταγλωττίσει (recompile) και να κάνει επανεκκίνηση το σύστημά του ώστε να δει τις αλλαγές του.

Σε αντίθεση, όμως, με τα Linux, το Skyl-OS είναι μικρό λειτουργικό σύστημα, ώστε να προσφέρεται για εκπαιδευτικές δραστηριότητες.
Για αυτό τον λόγο, το Skyl-OS δεν λειτουργεί με modules. Συνεπώς, πρέπει να ακολουθήσουμε τις οδηγίες, ώστε να κάνουμε compile 
τον πυρήνα του λειτουργικού. 

\begin{warn}[Προσοχή]
    Τα εργαλεία που χρησιμοποιούμε για να κάνουμε compile το Skyl-OS είναι γραμμένα για Linux. Προτείνεται η υλοποίηση όσων
    ζητούνται στη δραστηριότητα να γίνει σε Debian διανομή των Linux.
\end{warn}

\subsection{Τί είναι ένα Interrupt;}

Κατά τη λειτουργία του, ένα σύστημα συχνά δέχεται εξωτερικά, ή και εσωτερικά όπως θα δούμε, σήματα εισόδου. Όμως, όταν αυτά τα σήματα
έρθουν, ο επεξεργαστής θα διεξάγει κάποια άλλη διαδικασία. Ο ρόλος των Interrupts είναι, όπως φαίνεται και από το όνομά τους, να διακόψουν
τη λειτουργία του επεξεργαστή, χωρίς να επηρεάσουν την τρέχουσα κατάστασή του, ώστε αυτός να διαχειριστεί το σήμα εισόδου πριν συνεχίσει να
εκτελεί τη διεργασία που είχε αφήσει στη μέση.

\subsection{Είδη Interrupts}

Τα Interrupts χωρίζονται σε δύο μεγάλες κατηγορίες, τα Software Interrupts και τα Hardware Interrupts.

\subsubsection{Hardware Interrupts}

Τα Hardware Interrupts προέρχονται από εξωτερικές συσκευές (περιφερειακές), όπως το πληκρολόγιο, το ποντίκι, σκληροί δίσκοι ή κάρτες δικτύου.
Αυτά στέλνουν Interrupt Requests (IRQs), δηλαδή σήματα τα οποία προσδιορίζουν ποιος είναι ο στόχος του Interrupt. Για παράδειγμα, κάθε φορά
που ένα πλήκτρο πατιέται αυτό στέλνει ένα IRQ με συγκεκριμένη κωδικοποίηση στον επεξεργαστή (στην πραγματικότητα μέσω του PIC για το οποίο
μπορείτε να διαβάσετε περισσότερα \href{https://wiki.osdev.org/8259_PIC}{εδώ}). Έπειτα, αυτό το σήμα μεταφράζεται από το PIC σε ένα interrupt vector. Τέλος, είναι δουλειά του 
λειτουργικού συστήματος να αντιστοιχίσει το interrupt vector σε κάποιο ISR.

Στην x86 αρχιτεκτονική, μάλιστα, υπάρχουν κάποια προβλήματα τα οποία έχουν δημιουργηθεί από τη συνεργασία της IBM και της Intel. Όταν η IBM
έφτιαξε τους πρώτους υπολογιστές οι οποίοι έτρεχαν σε Real Mode, έφτιαξε το PIC ώστε τα IRQs 0-7 να αντιστοιχίζονται στα interrupt vectors
0x08-0x0F. Αργότερα, όταν η Intel χρησιμοποίησε το ίδιο υλικό για υπολογιστές σε Protected Mode, δέσμευσε τα vectors 0x00-0x1F για CPU 
Exceptions. Για να λύσουμε αυτό το πρόβλημα πρέπει να κάνουμε επαναχαρτογράφηση του PIC, ώστε τα IRQs 0-7 να αντιστοιχίζονται στα vectors
0x20 και μετά.

Τα Hardware Interrupts χωρίζονται και αυτά με τη σειρά τους σε Maskable και Non-Maskable. Τα πρώτα μπορούν να απενεργοποιηθούν, ή να αγνοηθούν,
προκειμένου να μή διακόψουν διεργασίες υψηλότερης προτεραιότητας, ενώ τα δεύτερα σημαίνουν ότι έχει συμβεί κάτι καταστροφικό (π.χ. πρόβλημα με
το υλικό του υπολογιστή).

\subsubsection{Software Interrupts}

Τα Software Interrupts προέρχονται από διαδικασίες εντός του λειτουργικού συστήματος, δηλαδή από προγράμματα (user or kernel space) ή από
exceptions (π.χ. πρόσβαση σε απαγορευμένη μνήμη).

Για να προκαλέσει ένα πρόγραμμα κάποιο Interrupt, πρέπει να κάνει μία κλήση στο λειτουργικό σύστημα (System Call), ώστε να ζητήσει κάποια 
υπηρεσία του συστήματος (π.χ. είσοδο/έξοδο, δέσμευση μνήμης). Οι κλήσεις λειτουργικών συστημάτων θα καλυφθούν σε μετέπειτα κεφάλαια. 

Εκτός, από προγραμματισμένα Interrupts, όμως, μπορεί να προκύψει και κάποιο πρόβλημα (exception) κατά την εκτέλεση κάποιου προγράμματος,
το οποίο δεν είναι αναμενόμενο. Για παράδειγμα, κάποιο πρόγραμμα μπορεί να προσπαθήσει να διαβάσει μνήμη η οποία δεν του ανήκει. Σε τέτοιες
περιπτώσεις ενεργοποιούνται συγκεκριμένα Interrupts αυτόματα από τον επεξεργαστή και το λειτουργικό μας σύστημα πρέπει να έχει μεριμνήσει, ώστε
να τα διαχειρίζεται σωστά.

\subsection{Πώς διαχειριζόμαστε τα Interrups}

Αρχικά, ο επεξεργαστής αναγνωρίζει ένα σήμα Interrupt. Μόλις γίνει αυτό ξεκινάει μία διαδικασία η οποία ονομάζεται context switching.
Στο context switching το λειτουργικό σύστημα πρέπει να αποθηκεύσει την τρέχουσα κατάσταση του επεξεργαστή, δηλαδή όλες τις τιμές των
register του και του Program Counter, συνήθως κάνοντας τες push στο stack.

Έπειτα, ο επεξεργαστής πρέπει να ξεκινήσει να εκτελεί μία ρουτίνα, ώστε να εξυπηρετήσει το Interrupts. Αυτές οι ρουτίνες ονομάζονται
Interrupt Service Routines (ISR). Για να γίνει η αντιστοίχιση του IRQ στο σωστό ISR, κάθε σήμα Interrupt έχει και ένα vector το οποίο
προσδιορίζει ποιά ρουτίνα πρέπει να κληθεί. Αυτή η αντιστοίχιση γίνεται μέσω ενός πίνακα, του Interrupt Vector Table (IVT) ή Interrupt
Descriptor Table (IDT). Ο επεξεργαστής έτσι μεταφέρει την εκτέλεση στο ISR και εξυπηρετεί το IRQ. 

Όταν τελειώσει η εκτέλεση του ISR, ο επεξεργαστής πρέπει να επιστρέψει στην κανονική του λειτουργία, συνήθως μέσω της εντολής iret.
Το λειτουργικό σύστημα επαναφέρει τις τιμές των καταχωρητών του επεξεργαστή από το stack και συνεχίζει την εκτέλεση του προηγούμενου 
προγράμματος από το σημείο που είχε σταματήσει.

\subsection{Interrupt Descriptor Table}

Όπως αναφέρθηκε και νωρίτερα, το IDT είναι ένας πίνακας που χρησιμοποιεί ο x86 επεξεργαστής, για να συνδέει κάθε IRQ με το αντίστοιχο
ISR του.

\subsubsection{Δομή του IDT}

Ο πίνακας αυτός έχει 256 εισαγωγές, κάθε μία εκ των οποίων έχει ένα interrupt vector number από το 0 έως και το 255. Κάθε εισαγωγή του
πίνακα (Gate Descriptor) κρατάει τις εξής πληροφορίες: τον segment selector που δείχνει σε ποιό code segment βρίσκεται ISR, το offset, 
δηλαδή τη θέση του ISR μέσα στο code segment, το ελάχιστα δικαιώματα που πρέπει να έχει ένα πρόγραμμα για να εκτελεστεί το ISR (Descriptor
Privilege Level ή DPL), και το είδος του gate, δηλάδή αν είναι hardware interrupt ή software interrupt. Περιέχει,
δηλαδή, τις απαραίτητες πληροφορίες για να εκτελεστούν ρουτίνες υψηλών privileges μέσω κώδικα χαμηλών privileges.

Το IDT μπορεί να έχει διαφορετικό μέγεθος ανάλογα με το πλήθος των interrupts που περιέχει και μπορεί να βρίσκεται οπουδήποτε στη φυσική
μνήμη. Για να έχει πρόσβαση ο επεξεργαστής σε αυτό πρέπει το λειτουργικό σύστημα να προσδιορίσει που βρίσκεται το IDT. Για να το κάνει
αυτό χρησιμοποιεί ένα 48-bit register του επεξεργαστή το IDT Register. Μέσα σε αυτό βρίσκεται η διεύθυνση στην οποία αρχίζει το IDT και
το όριο του.

\begin{figure}[H]
    \centering
    \begin{tabular}{|c|c|l|l|}
        \hline
        \textbf{Vector} & \textbf{Type} & \textbf{Description} & \textbf{Handler Function} \\
        \hline
        \texttt{0x00} & Exception & Divide by Zero & \texttt{isr\_32} \\
        \hline
        \dots & \dots & \dots & \dots \\
        \hline
        \texttt{0x01} & Exception & Breakpoint & \texttt{isr\_3} \\
        \hline
        \dots & \dots & \dots & \dots \\
        \hline
        \texttt{0x0E} & Exception & Page Fault & \texttt{isr\_14} \\
        \hline
        \dots & \dots & \dots & \dots \\
        \hline
        \texttt{0x20} & IRQ 0 & Timer (PIT) & \texttt{isr\_0} \\
        \hline
        \texttt{0x21} & IRQ 1 & \textbf{Keyboard} & \texttt{isr\_1} \\
        \hline
        \dots & \dots & \dots & \dots \\
        \hline
    \end{tabular}
    \caption{Η χαρτογράφηση του IDT στο Skyl-OS. Οι πρώτες 32 θέσεις είναι δεσμευμένες από την Intel (Exceptions), ενώ οι θέσεις 0x20-0x2F χρησιμοποιούνται από τον επαναχαρτογραφημένο PIC.}
    \label{fig:idt_map}
\end{figure}

\subsubsection{Λειτουργία του IDT}

Όταν ο επεξεργαστής λαμβάνει ένα σήμα interrupt ανατρέχει στο IDT ώστε να προσδιορίσει τη θέση στη μνήμη του ISR. Το σήμα που λαμβάνει
περιέχει και το vector number του interrupt. Χρησιμοποιώντας τον παρακάτω υπολογισμό ο επεξεργαστής μπορεί να προσδιορίσει τη θέση του
ISR: $$\text{ISR Address} = \text{IDTR.Base} + (\text{Vector Number} \times \text{Descriptor Size})$$

Έπειτα, ο επεξεργαστής ελέγχει ότι το τρέχον privilege level είναι μεγαλύτερο ή ίσο του DPL του gate και μεταφέρει την εκτέλεση στο ISR.
Αν το gate δείχνει ότι είναι hardware interrupt ο επεξεργαστής αυτόματα μηδενίζει το IF, ώστε να απενεργοποιήσει τα maskable interrupts.
Αν το gate δείχνει ότι είναι software interrupt ο επεξεργαστής αφήνει το IF ενεργό, αφήνοντας άλλα interrupts να καλούνται, καθώς τα
software interrupts μπορεί να είναι χρονοβόρα. Έτσι, εξασφαλίζεται ότι το σύστημα εξυπηρετεί όλες τις ανάγκες των interrupts.

\begin{figure}[H]
    \centering
    \includegraphics[width=0.8\textwidth]{images/Interrupt_Figure.png}
    \caption{Interrupt Handling Flow Visual}
    \label{fig:int_handling}
\end{figure}

\subsection{Προετοιμασία περιβάλλοντος ανάπτυξης.}

Όπως αναφέρθηκε νωρίτερα το περιβάλλον στο οποίο γίνεται η ανάπτυξη του Skyl-OS είναι Debian Linux. Αν το κύριο λειτουργικό σας σύστημα δεν
είναι Debian Linux προτείνεται να δουλέψετε σε κάποιο εικονικό μηχάνημα (virtual machine), ώστε να μπορείτε να εγκαταστήσετε να απαραίτητα
εργαλεία.

Για να έχετε ένα εικονικό μηχάνημα μπορείτε να χρησιμοποιήσετε το δωρεάν λογισμικό \href{https://www.virtualbox.org/}{Virtual Box}
Από εκεί, μπορείτε να δημιουργήσετε ένα εικονικό μηχάνημα χρησιμοποιώντας οποιαδήποτε διανομή των Linux επιθυμείτε.  
Διαφοροποιήσεις μεταξύ διανομών θα υπάρχουν μόνο στη διαδικασία απόκτησης των απαραίτητων πακέτων για την ανάπτυξη του κώδικα, όμως,
σε αυτό τον οδηγό δίνονται τα πακέτα για Debian Linux.

\subsubsection{Απαραίτητα Πακέτα}

Θα χρειαστεί να υπάρχουν στο σύστημά σας ορισμένα πακέτα:

Για διανομές βασισμένες στα Debian:

\begin{commandline}
    \begin{verbatim}
$ sudo apt-get update
$ sudo apt install build-essential nasm qemu-system-x86
    \end{verbatim}
\end{commandline}

\subsubsection{Λήψη του πηγαίου κώδικα}

Ο πηγαίος κώδικας του Skyl-OS βρίσκεται στο GitHub. Για να τον επεξεργαστείτε πρέπει πρώτα να τον κάνετε clone από το repository στο
μηχάνημά σας και να αλλάξετε στο branch idt-lab στο οποίο υπάρχει ο κώδικας της άσκησης.

\begin{commandline}
    \begin{verbatim}
$ git clone https://github.com/Billyzeim/Skyl-OS.git
$ git checkout idt-lab
    \end{verbatim}
\end{commandline}

\begin{info}[Σημείωση]
    Το branch idt-lab έχει φτιαχτεί για διδακτικούς σκοπούς και εσκεμμένα έχει κάποια errors. Στις παρακάτω δραστηριότητες θα κληθείτε
    να φτιάξετε τα εν λόγω errors.
\end{info}

\subsubsection{Compilation και εκτέλεση}

Για να κάνετε compile και να τρέξετε το Skyl-OS στο qemu virtual machine χρησιμοποιήστε την παρακάτω εντολή:

\begin{commandline}
    \begin{verbatim}
$ make run
    \end{verbatim}
\end{commandline}

\section{Υλοποίηση IDT στο Skyl-OS}

Σε αυτή την ενότητα θα εργαστούμε πάνω στον κώδικα του Skyl-OS για την αρχικοποίηση του Interrupt Descriptor Table (IDT).

\begin{question}    
    Στόχος της άσκησης είναι η κατανόηση της δομής μιας IDT entry (gate descriptor) και του τρόπου με τον οποίο ο x86 επεξεργαστής 
    μεταβαίνει από κανονική εκτέλεση σε έναν interrupt handler. Επίσης, θα δούμε πως οι αρχιτεκτονικές αποφάσεις τον μηχανικών εισάγουν 
    περιορισμούς στα συστήματά μας τους οποίους καλούμαστε να ξεπεράσουμε.

    Μελετήστε το αρχείο \src{idt\_implementation.c}. Εστιάστε στη συνάρτηση \src{helper} και στα κενά που πρέπει να συμπληρωθούν στην \src{idt\_init}.

    \begin{file}[idt\_implementation.c (Snippet)]
    \footnotesize \begin{lstlisting}[language=C, tabsize=2]
void helper(uint8_t num, uint32_t base, 
        uint16_t sel, uint8_t flags) {
    idt_entries[num].base_low  = base & 0xFFFF;
    idt_entries[num].base_high = (base >> 16) & 0xFFFF;
    idt_entries[num].sel       = sel;
    idt_entries[num].flags     = flags;
    // ...
}
    \end{lstlisting}
    \end{file}

    \begin{file}[idt\_implementation.c (Snippet)]
        \footnotesize \begin{lstlisting}[language=C, tabsize=2]
helper(0, (uint32_t) _________, 0x08, 0x8E);
    \end{lstlisting}
    \end{file}

    \begin{file}[idt.h (Snippet)]
        \footnotesize \begin{lstlisting}[language=C, tabsize=2]
// 1. The IDT Entry Structure
struct idt_entry_struct {
    uint16_t base_low; // The lower 16 bits of
                       // the address to jump to
    uint16_t sel; // Kernel segment selector
    uint8_t  always0; // This must always be 0
    uint8_t  flags; // More flags (Present, DPL, etc)
    uint16_t base_high; // The upper 16 bits of the 
                        // address to jump to
} __attribute__((packed));
    \end{lstlisting}
    \end{file}

    \textbf{Ζητούμενα:}
    
    \begin{enumerate}
        \item Αναλύστε τη δομή μιας IDT entry (gate descriptor) από το αρχείο \src{idt.h}.
        \begin{enumerate}
            \item Τι αναπαριστά κάθε πεδίο της δομής;
            \item Γιατί η διεύθυνση του ISR (\src{base}) χωρίζεται σε base\_low και \src{base\_high};
            \item Γιατί το πεδίο \src{always0} πρέπει να είναι μηδέν; Τι θα συμβεί αν δεν είναι;
            \item Τι πληροφορία περιέχει το πεδίο flags; Ποια bits καθορίζουν:
            \begin{enumerate}
                \item αν το interrupt είναι ενεργό;
                \item το privilege level;
                \item τον τύπο του gate;
            \end{enumerate}
        \end{enumerate}
        \item Ποιός είναι ο ρόλος της συνάρτησης \src{helper}; Σε τι χρησιμεύει; Δώστε της ένα περιγραφικό όνομα.
        \item Συμπληρώστε τα κενά (\_\_\_\_\_\_\_\_) στον κώδικα ώστε να οριστούν οι handlers για Division by Zero, Page Fault και Breakpoint. Δείτε το αρχείο idt.asm για βοήθεια στις ονομασίες.
        \item Γιατί είναι κρίσιμο να απενεργοποιήσουμε τα interrupts (\src{cli}) στην αρχή; Τι θα συμβεί αν ο Timer (IRQ0) στείλει σήμα πριν φορτωθεί το IDT;
        \item Τι είναι ο PIC και γιατί πρέπει να τον επαναχαρτογραφήσουμε (remap) στις διευθύνσεις \src{0x20 - 0x2F}; 
    \end{enumerate}

\end{question}

\begin{question}
    Ένα κρίσιμο κομμάτι της διαχείρισης διακοπών είναι η σωστή διαχείριση της στοίβας (stack). Όταν συμβαίνει μια διακοπή, 
    ο επεξεργαστής και ο assembly wrapper σπρώχνουν τιμές στη στοίβα που περιγράφουν την κατάσταση του συστήματος.

    Παρακάτω δίνεται ο κώδικας για τον χειριστή του Breakpoint Exception (INT 3) στο αρχείο \src{isrs.c} και το assembly stub με το οποίο
    καλείται η ρουτίνα isr\_3.

    \begin{file}[isr\_3.c]
        \footnotesize \begin{lstlisting}[language=C, tabsize=2]
#include <stdint.h>
#include <terminal_io.h>
#include <isrs.h>

typedef struct {
    // Pushed by pushad
    uint32_t edi, esi, ebp, esp, ebx, edx, ecx, eax; 
    // Pushed by CPU
    uint32_t eip, cs, eflags;                        
} registers_t;

void isr_3(registers_t *regs) {
    printf("\n=== DEBUG BREAKPOINT ===\n");
    kprint("EAX: "); printf("%x\n", regs->eax);
    kprint(" EBX: "); printf("%x\n", regs->ebx);
    kprint(" EIP: "); printf("%x\n", regs->eip);
    kprint("System Halted for inspection.\n");
    printf("=== DEBUG BREAKPOINT END ===\n");
}
        \end{lstlisting}
    \end{file}
    
    \begin{file}[idt.asm]
        \footnotesize \begin{lstlisting}[language={[x86masm]Assembler}, tabsize=2]
isr_3_asm:
    pushad
    push esp ; This passes a pointer
    ; for the regs structure
    call isr_3  ; calls void isr_3
    add esp, 4 ; Clean up the stack
    popad ; Now loads from stack
    iret
        \end{lstlisting}
    \end{file}

    \textbf{Ζητούμενα:}

    \begin{enumerate}
        \item Παρατηρήστε ότι περνάμε τη δομή \src{registers\_t} μέσω δείκτη. Ποιός είναι αυτός ο δείκτης; Γιατί δεν αντιγράφουμε απλώς τις τιμές των registers στη στοίβα;
        \item Αν θέλατε να τροποποιήσετε την τιμή του καταχωρητή \src{EAX} ώστε το πρόγραμμα να συνεχίσει με νέα τιμή μετά την επιστροφή από το interrupt, πώς θα το κάνατε μέσα από την \src{isr\_3};
        \item Αν περνούσαμε τις τιμές μέσω αντιγραφής θα μπορούσαμε να τροποποιήσουμε καταχωρητές με ανάλογο τρόπο; Δοκιμάστε το!
    \end{enumerate}

\end{question}

\newpage

\section{Δημιουργία Οδηγού Πληκτρολογίου (Keyboard ISR)}

Σε αυτή την άσκηση θα υλοποιήσετε έναν πραγματικό οδηγό υλικού (Device Driver). Θα δημιουργήσετε έναν ISR που θα εκτελείται κάθε φορά που πατάτε ένα πλήκτρο.

Το πληκτρολόγιο συνδέεται στη γραμμή \textbf{IRQ 1} του PIC. Επειδή έχουμε επαναχαρτογραφήσει τον PIC στο \src{0x20}, το IRQ 1 αντιστοιχεί στο \textbf{Vector 0x21} του IDT.
Για να διαβάσουμε ποιο πλήκτρο πατήθηκε, πρέπει να διαβάσουμε 1 byte από τη θύρα δεδομένων του πληκτρολογίου (\src{0x60}).

\begin{question}
    Παρακάτω δίνεται ο σκελετός του \src{isr\_1} στο αρχείο \src{keyboard\_isr.c}.

    \begin{file}[keyboard\_isr.c]
    \footnotesize \begin{lstlisting}[language=C, tabsize=2]
void isr_1() {
    uint8_t scancode = _______; 
    // Use the inb function. The keyboard input is at 0x60

    // check if key was pressed using !(scancode & 0x80) 
    // and ignore key releases
    // convert scancode to ascii 
    // using scancode_to_ascii function
    // print it with printf
}
    \end{lstlisting}
    \end{file}

    \textbf{Ζητούμενα:}

    \begin{enumerate}
        \item Στο αρχείο \src{idt.c}, προσθέστε την κατάλληλη πύλη (gate) για το \textbf{Vector 0x21} ώστε να καλεί τον \src{irq\_stub\_1} (ο οποίος με τη σειρά του καλεί την \src{isr\_1}).
        \item Συμπληρώστε τον κώδικα της \src{isr\_1}:
        \begin{itemize}
            \item Χρησιμοποιήστε την εντολή \src{inb(0x60)} για να διαβάσετε τον κωδικό (scancode).
        \end{itemize}
        \item Κάντε Compile, τρέξτε το λειτουργικό και πατήστε διάφορα πλήκτρα. Θα πρέπει να βλέπετε τους δεκαεξαδικούς κωδικούς (scancodes) και τους αντίστοιχους χαρακτήρες στην οθόνη.
    \end{enumerate}

\end{question}

\section{"Αγγίζοντας" τις κλήσεις λειτουργικού}

Σε αυτή την άσκηση θα φτιάξουμε μία κλήση λειτουργικού συστήματος. Η ανάπτυξη πλήρως λειτουργικών κλήσεων λειτουργικού δεν θα μας απασχολήσει σε αυτή τη δραστηριότητα, όμως
είναι σημαντικό, όμως, να συνειδητοποιήσουμε τη σύνδεσή τους με τα interrupts.

\begin{question}
    Σε αυτή την άσκηση θα υλοποιήσετε ένα syscall από το μηδέν. Το stub της assembly είναι έτοιμο υλοποιημένο.

    Για να το δείτε να δουλεύει καλέστε τη συνάρτηση \src{syscall\_echo} στον φάκελο \src{kernel.c}.

    \textbf{Ζητούμενα:}

    \begin{enumerate}
        \item Συμπληρώστε τον κώδικα του service routine για την κλήση του λειτουργικού που ζητείται, ώστε να εκτυπώνεται η τιμή του \src{EAX} σε χαρακτήρα.
        \item Κάντε map το service routine στο vector 0x80 του IDT.
        \item Υλοποιήστε μία συνάρτηση \src{print\_with\_syscall} μέσα στο αρχείο \src{kernel.c} η οποία θα δέχεται ως όρισμα έναν δείκτη σε χαρακτήρα και θα εκτυπώνει
        το κείμενο χρησιμοποιώντας το \\ παραπάνω syscall.
    \end{enumerate}

    Για τυχόν απορίες ανατρέξτε στις παραπάνω ασκήσεις.
\end{question}

\section{Debugging στα τυφλά (Bonus)}

Σε αυτή την ενότητα θα μάθουμε να σκεφτόμαστε από το μηδέν σαν μηχανικοί λειτουργικών συστημάτων. 

Η μεγαλύτερη δυσκολία στην ανάπτυξη λειτουργικών συστημάτων είναι η μικρή απόσταση με το υλικό του υπολογιστή. Η συνεργασία κώδικα και επεξεργαστή, χωρίς ένα
ήδη υπάρχον στώρμα αφαιρετικότητας από κάποιο λειτουργικό σύστημα καθιστά το debugging δύσκολο. Σε αυτή την άσκηση θα προκαλέσουμε ένα παράδοξο πρόβλημα και...
θα κολυμπήσουμε στα βαθιά. Θα προσπαθήσουμε να εξηγήσουμε το πρόβλημα μέσω debugging και γνώσεων που μάθαμε έως τώρα.

\begin{question}
    Για να αναπαράξετε το bug στον κώδικα σβήστε τον βρόχο while που υπάρχει στο τέλος της \src{kernel.c}, χωρίς να σβήσετε την εντολή \src{halt}.
    Έπειτα, αξιοποιήστε τη συνάρτηση \src{mask\_timer}, ώστε να \\ απενεργοποιήσετε το interrupt του timer, για να μην ενοχλεί στο debugging.

    \textbf{Ζητούμενα:}

    \begin{enumerate}
        \item Προκαλέστε κάποιο interrupt. Τι συμβαίνει; Γιατί; Πώς λύνεται αυτό το πρόβλημα και γιατί;
    \end{enumerate}

    Καλή τύχη!
\end{question}

\section{Παραδοτέα}

Συντάξτε μία αναφορά σε \src{pdf} η οποία θα περιέχει τις απαντήσεις στις θεωρητικές ερωτήσεις (και όχι μεγάλα κομμάτια κώδικα), τα σημεία του κώδικα τα οποία έχετε αλλάξει
και τις επεξηγήσεις των αλλαγών. Το υποβληθέν θα είναι ένα συμπιεσμένο αρχείο \src{zip} με όνομα upAM.zip με τα αρχεία κώδικα τα οποία έχετε τροποποιήσει.
Καλή επιτυχία!

\section*{Περισσότερες Πηγές}

\begin{itemize}
    \item \href{https://wiki.osdev.org/Interrupt_Descriptor_Table}{OSDev Wiki - IDT}
    \item \href{https://wiki.osdev.org/8259_PIC}{OSDev Wiki - 8259 PIC}
    \item \href{https://www.doc-developpement-durable.org/file/Projets-informatiques/cours-&-manuels-informatiques/Linux/Linux\%20Kernel\%20Development,\%203rd\%20Edition.pdf}{Linux Kernel Development}
    \item Operating System Concepts 9th Edition - Silbershatz, Galvin, Gagne
    \item \href{https://github.com/dreamportdev/Osdev-Notes/blob/master/02_Architecture/05_InterruptHandling.md}{Osdev-Notes}
\end{itemize}

\end{document}